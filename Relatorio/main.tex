\documentclass[12pt]{article} 

\usepackage{geometry} 
\geometry{a4paper} 
\usepackage{graphicx} 
\usepackage{float} 
\usepackage{wrapfig} 
\linespread{1.2}
\graphicspath{{Pictures/}} 
\usepackage[utf8]{inputenc}
\usepackage[T1]{fontenc}
\usepackage[portuguese]{babel}

\begin{document}

%----------------------------------------------------------------------------------------
%	TITLE PAGE
%----------------------------------------------------------------------------------------

\begin{titlepage}

\newcommand{\HRule}{\rule{\linewidth}{0.5mm}} 
\center
\begin{figure}[ht!]
	\centering
	\includegraphics[scale=3]{DI.png}
\end{figure}

\textsc{\large Sistemas Operativos}\\[0.5cm] 
\HRule \\[0.4cm]
{ \huge \bfseries Stream Processing}\\[0.4cm] 
\HRule \\[1.5cm]


\begin{center} \large
\emph{Grupo 19:}\\
José Martins - a78821\\
Madalena Castro - a71417\\
Miguel Quaresma - a77049\\
\end{center}

\begin{minipage}{0.4\textwidth}
\end{minipage}\\[4cm]

{\large \today}\\[3cm] 
\vfill
\end{titlepage}

%----------------------------------------------------------------------------------------
%	TABLE OF CONTENTS
%----------------------------------------------------------------------------------------

\tableofcontents 

\newpage

%----------------------------------------------------------------------------------------
%	INTRODUÇÃO
%----------------------------------------------------------------------------------------

\section{Introdução}
\paragraph{}
\textit{Stream Processing} é uma arquitetura computacional que tem como intuito facilitar o uso de paralelismo no processamento de dados ao abstrair o utilizador da gestão dos recursos disponíveis (i.e: comunicação entre nodos, uso de recursos,etc). Os dados são recebidos sob a forma de streams e operações que são aplicadas a cada elemento da stream. 

A linguagem de programação C disponibiliza um conjunto de chamadas ao sistema que permitem a exploração de paralelismo simples, sendo o intuito deste trabalho recorrer a estas mesmas chamadas para implementar um sistema similar ao referido.


\newpage

%----------------------------------------------------------------------------------------
%	DESENVOLVIMENTO
%----------------------------------------------------------------------------------------

\section{Desenvolvimento} 

\subsection{Estrutura}
\begin{figure}[ht!]
\centering
\includegraphics[width=120mm]{graph.jpg}
\end{figure}

\subsubsection{Controlador}
O controlador é a entidade responsável por processar input e gerir a rede de processamento, nomeadamente através da criação/remoção de nodos e gestão das ligações entre os mesmos.
A gestão dos nodos recorre a três arrays para manter informação relativa às ligações entre os nodos, ao pid(process id) dos mesmos e aos descriptores dos named pipes que servem de input para os nodos.
A leitura do input é feita linha a linha através de um named pipe com o nome \textit{input}, de seguida o input é classificado como instruções ou dados através da presença(ou ausência) do caracter ':'.
Os comandos disponíveis são os seguintes:

\begin{description} 
\item[newNode] \hfill \\
   Função que realiza o comando \texttt{node <id> <cmd> <args...>}, definindo um nodo de rede de processamento, com o identificador  \texttt{id}, que executará o componente \texttt{cmd} com a lista de argumentos \texttt{[args]}. Para tal,começamos por criar um pipe com nome com o nome "node<id>", sendo que na posição do id estará o número do nodo, este pipe com nome como já indicado irá servir para o nodo receber input bem como comandos do controlador. De seguida tratamos é criado um nodo que irá tomar o papel de supervisor. Por fim, abrimos o pipe com nome, guardando o \textit{file descriptor} do mesmo num array bem como o \textit{process ID} necessário para posteriores ações sobre o supervisor.


\item[connect] \hfill \\
    Função que realiza o comando \texttt{connect <id> <ids...>}, que permite definir ligações entre nodos, ou seja, especifica que a saída do componente a correr no nodo \texttt{id} deverá ser enviada para cada um dos nodos da lista \texttt{[ids]}. Nesta função, atualizamos o status(array com tamanho igual ou maior ao número de nodos em que em cada posição possui uma lista com a informação de que nodos recebe input) dos nodos que irão passar a receber o input do nodo presente como 1º argumento. Contudo para o nodo que envia input enviamos-lhe o comando ";cid" em que id é trocado pelo valor dos vários nodos que irão receber input, de modo a que saiba a quem enviar input.

\item[inject] \hfill \\
    Função que realiza o comando \texttt{inject <id> <cmd> <args...>}, que permite injetar na entrada do nodo \texttt{id} a saída produzida pelo comando cmd executado com a lista de argumentos \texttt{[args]}. Com a nossa implementação esta função é bastante simples bastando redirecionar o \textit{stdout} para o pipe com nome correspondente ao passado como argumento e após isso correr o comando. 

\item[disconnect] \hfill \\
    Função que realiza o comando \texttt{disconnect <id1> <id2>}, que permite desfazer a ligação entre o nodo \texttt{id1} e o nodo \texttt{id2}. Nesta função atualizamos o valor status do nodo id2 (de modo a saber que não recebe mais input do nodo id1) e, após isso mandamos o comando '';d<id2>'' ao nodo id1 de maneira a saber que não deve mais enviar input ao nodo id2. 

\item[removeNode] \hfill \\
    Função que realiza o comando \texttt{remove <id>}, que permite a remoção de um nodo na rede. Em primeiro lugar, removemos a ligação entre os nodos que enviam input para o nodo a ser removido e, ao mesmo tempo ligamos esses nodos com os nodos que recebiam input do nodo a ser removido. De seguida, removemos a ligação entre o nodo a ser removido e os nodos que recebiam input dele. Por fim, fecha-se o pipe com nome do nodo a ser removido, remove-se esse mesmo pipe, enviamos um sinal de interrupção ao supervisor e atualizamos o array dos \textit{process ID} colocando o valor a 0, de modo a garantir que não é enviado input para este nodo.
\end{description}

Determinados comandos exigem ações por parte dos nodos e, por forma a que estes sejam diferentes dos dados enviados para os mesmos(nodos) é usado como prefixo o caracter ';' seguido de um caracter que identifique o comando a executar ('c' para \textit{connect} e 'd' para \textit{disconnect}).

\subsubsection{Supervisor}
Os nodos são constitúidos por três entidades principais, sendo que o supervisor é responsável pela criação destas entidades.
O supervisor, apesar de não participar no processamento dos dados, é responsável pela criação do nodo de processamento em si, bem como pela criação de dois processos que irão executar duas funções: 
\begin{itemize}
\item \texttt{manInput}:gestão do input recebido, fazendo a distinção entre dados e comandos, enviando o mesmo para o nodo ou para o outro filho
\item \texttt{manOutput}:distribuição do output do nodo pelos nodos aos quais se encontra ligado bem como gestão dos file descriptors para os quais tem que enviar input
\end{itemize}
O output predefinido de um nodo é /dev/null, sendo que a partir do momento em que este(nodo) é ligado a outro o output é redireccionado para o named pipe que serve de input ao mesmo. A sintaxe usada na atribuição dos nomes dos \textit{pipes}é a seguinte "node<id>", sendo que id é o identificador do nodo, ou seja, o nodo 1 terá como input o \textit{named pipe} "node1"

\newpage
\subsection{Componentes} 
\begin{description} 
\item[Const] \hfill \\
O componente \textit{const} reproduz as linhas recebidas acrescentando uma nova coluna sempre com o mesmo valor. Para isso, lemos caractér a caractér recebido pelo \textit{stdin} e após verificar que estamos no término de uma linha ('\textbackslash n'), adicionamos uma nova coluna com o valor dado como argumento. Posteriormente, concluímos a linha com '\textbackslash n', e reproduzimo-la no \textit{stdout}, realizando esta operação para todas as linhas recebidas. 

\item[Filter] \hfill \\
O componente \textit{filter} reproduz as linhas que satisfazem uma condição passada como argumento. Esta condição é composta por um operador de comparação e dois valores que correspondem a colunas das linhas a ser processadas. Inicialmente começamos por dividir a linha recebida pelo \textit{stdin} consoante o \textit{token} ':',caso estejamos numa linha presente na condição guardamos o valor numa variavel local, para posteriormente verificarmos a condição recebida, caso verdadeira a linha será reproduzida no \textit{stdout}, caso falso será desprezada.

\item[Window] \hfill \\
O componente \textit{window} reproduz todas as linhas acrescentando-lhe uma nova coluna com o resultado de uma operação calculada sobre valores da coluna indicada nas linhas anteriores. Inicialmente começamos por verificar os argumentos recebidos, guardando a coluna, a operação e quais as linhas anteriores a serem calculadas. De seguida, começamos a leitura das linhas recebidas pelo \textit{stdin} e à medida que uma linha termina é reproduzido no \textit{stdout} a linha com a coluna que representa o cálculo da operação, previamente calculada consoante o tipo de operação, ou seja, inicialmente, como não há linhas anteriores a nova coluna será zero, mas à medida que as linhas vão sendo lidas, o valor será recalculado, tendo em conta as novas colunas, e guardado numa variável que será posteriormente utilizada para criar a nova coluna. Consoante o tipo de operação enviada como argumento, são chamadas diferentes funções.

\item[Spawn] \hfill \\
    O componente \textit{spawn} reproduz todas as linhas, executando o comando indicado uma vez para cada uma delas, e acrescentando uma nova coluna com o respectivo \textit{exit status}. É permitido o uso de \textit{'\$n'} no qual é indicado o número de uma coluna das linhas, de modo a que ao executar o comando, nos seus argumentos na posição do \textit{'\$n'} esteja o conteúdo presente na coluna n de cada linha. Para tal, vamos lendo a linha, caracter a caracter, e quando encontramos a/s coluna/s especificada/s guardamos o conteudo numa variavel local. Quando chegamos ao '\textbackslash n' executamos o argumento num filho, porém tendo o cuidado de substituir as partes com '\$n' pelo correto conteúdo.  

\end{description} 

\newpage
\subsection{Funcionalidades básicas}
\begin{description} 

\item[Concorrência nas escritas no mesmo nodo] \hfill \\
    Visto tratar-se de um sistema que explora paralelismo e visto ser possível ter vários nodos a escrever para o mesmo ficheiro (ex: pipe de input de um nodo) é necessário garantir que cada evento chega ao destino corrompido. Para garanti que este requisito é cumprido as escritas para os pipes/ficheiros são feitas em bloco, neste caso, linha a linha, visto que o sistema se baseia todo em eventos, um por linha. Por outro lado, para garantir que os comandos e dados chegam na ordem que foram enviados recorremos a sinais, sendo exemplo disto o uso dos sinais \texttt{SIGSTOP} e \texttt{SIGCONT} para sincronizar a chegada de input ao manOutput no supervisor.

\item[Nós sem saída definida] \hfill \\
Os nodos cuja saída não esteja ligada a nenhum outro devem ter a sua saída descartada, para isso esta é direccionada para /dev/null.

\item[Especificação incremental na rede] \hfill \\
O servidor vai recebenco comandos, devendo atuar perante cada um, atualizando a rede, independentemente de comandos futuros. O nosso programa está preparado para receber continuamente comandos, e atualizar constantemente a rede de nodos.

\end{description}

\newpage
\subsection{Funcionalidades adicionais}

\begin{description} 
\item[Casos de teste] \hfill \\
Para testar o nosso programa realizámos um caso de teste. Para isso criámos um ficheiro \texttt{.txt} com o seguinte exemplo:
\begin{verbatim}
    node 1 grep -w int filter.c
    node 2 grep -w int const.c
    node 3 grep -w int spawn.c
    node 4 grep -w int window.c
    node 5 grep -w int commands.c
    node 6 tee result.txt
    connect 1 6
    connect 2 6
    connect 3 6
    connect 4 6
    connect 5 6
\end{verbatim}
O objetivo deste caso é paralelizar a pesquisa por palavras em ficheiros, colocando o output na pesquisa num ficheiro txt. 

%\item[Alteração de componentes] \hfill \\

\end{description}

\newpage
%----------------------------------------------------------------------------------------
%	CONCLUSÃO
%----------------------------------------------------------------------------------------

\section{Conclusão} 
\paragraph{}
Neste trabalho, o principal objetivo era a utilização da linguagem de programação C, para a construção de um sistema de \textit{stream processing}, pretendendo a implementação de um conjunto de componentes que realizam tarefas elementares, e a implementação de um sistema que compõe e controla a rede de processamento.

De uma maneira geral conseguimos implementar todas as funcionalidades pedidas no enunciado, exceptuando a 'Alteração de Componentes', que permite a alteração do componente a correr num nodo. 

Uma das maiores dificuldades encontradas na implementação do projeto foi o controlo de concorrência de escritas no mesmo nodo, conseguido apenas em alguns casos de forma satisfatória. 


%----------------------------------------------------------------------------------------

\end{document}
